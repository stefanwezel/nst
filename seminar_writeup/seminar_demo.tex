%%%%%%%%%%%%%%%%%%%%%%%%%%%%%%%%%%%%%%%%%%%%%%%%%%%%%%%%%%%%%%%%%%%%%%%%%
%
% LaTeX-Vorlage fuer die Gestaltung der Ausarbeitung zu einem Seminar
%
% Basierend auf einer Vorlage von Joerg Willmann vom 06.06.2002
% Ueberarbeitet von Clemens Juergens am 12.06.2002
%
%%%%%%%%%%%%%%%%%%%%%%%%%%%%%%%%%%%%%%%%%%%%%%%%%%%%%%%%%%%%%%%%%%%%%%%%%


\documentclass{seminar}
%\usepackage[applemac]{inputenc}
\usepackage[T1]{fontenc}
\usepackage[latin1]{inputenc}
\usepackage[english]{babel}
\usepackage{graphicx}
%\usepackage{url}
\usepackage[round]{natbib}
\usepackage{algorithm2e}



\begin{document}
\renewcommand\toptitle{Seminar: ,,Multimedia-Standards im Internet``}
\title{Neural Style Transfer}
\author{Stefan Wezel}
\maketitle


\addvspace{0.5cm}
\emph{\bfseries{Abstract:}}
\emph{Please give a short version of the content of your report.}


\tableofcontents
\newpage

\section{Introduction}
Don't forget to change the head of your report. title, seminar, author, ...

\subsection{sub section}
FIGURES have captions below the figures!!!\\
TABLES have captions above the tables!!!\\

\section{Conclusion}
Don't forget to cite !!!\\
How to cite something: \cite{Subramanian2005, Bjoerkholm2009}.



  \begin{table}
  \caption{caption}
  \begin{tabular}{|c|c|c|}
\hline
 A & B & C \\
\hline
 1 & 2 & 3  \\
\hline 
 4 & 5 & 6 \\
\hline
\end{tabular}
  \end{table}


\newpage

% Variante mit seperatem Bibtex-File

\bibliographystyle{natbib}
\bibliography{seminarRefs}


% Variante zum manuellen Eintragen der Referenzen
%\begin{thebibliography}{01}
%\bibitem[Hunt98]{Hunt98}
%C. Hunt, TCP/IP Network Administration, O'Reilly 1998.
%\bibitem[Schnei94]{Schnei94}
%Dr. G. Schneider, Internet: Werkzeuge und Dienste, Springer-Verlag
%Berlin Heidelberg 1994.
%\end{thebibliography}
\end{document}



